\documentclass[12pt, letterpaper]{book}
\setlength{\parindent}{0pt}
\usepackage{graphicx}
\usepackage{caption}
\graphicspath{{images/}}
\title{Artificial Intelligence}
\author{Peter Jepson}
\date{November 2025}
\begin{document}
\begin{titlepage}
    \centering
    \vspace{1cm}
    {\Huge módcræft\par}
	\vspace*{0.5cm}
    {\Large Inside the Mind of a Machine \par}
	\vspace*{2.0cm}
    \includegraphics[width=0.75\textwidth]{nn} % nn.png
    \vspace{0.5cm}
	\newline

    \vspace{2cm}
    \vfill
    {\large Peter A Jepson\par}
    \vspace{0.2cm}
    {\large November 2025 \par}
\end{titlepage}
\maketitle
\tableofcontents
\newpage
\chapter{Hypothesis}
\section{Context}
\flushleft
The hypothesis of this book could be summarised as:
\begin{quote}
	\centering
	"How do we Build a Machine Mind?"
\end{quote}
\flushleft
Whilst this one line may sound intriguing, it does not adequately capture the scope and depth of the problem.
There are many questions; What is a Mind? What are the limitations of a Machine? etc...
\vspace{0.2cm}
\newline
You may now be expecting to read a Popular Science book, sweeping over the current progress of Artificial Intelligence. Alas, this will not be the case.
It is my firm belief that the current most up-to-date Artificial Intelligence implementations, the ominous Large Language Models, are not Minds. Therefore,
this book will be a serious attempt to lay a foundation of building a Mind with Computers so that we may indeed build one.
\vspace{0.2cm}
\newline
TODO: Continue this chapter as the book progresses

\chapter{A Computeable Theory of Mind}
\section{A Brief Tour of Electronic Computing Machines}
Before we investigate how we might model a Mind, first we must define the Machine and understand its nature. In our case, the Machine will be a General Purpose Electronic Computer.
This is obviously not the only kind of Machine and, perhaps to actually build a Mind, we will require more than just a Digital Computer. For now, we will assume it is
possible to build a Mind from an Electronic Computer until we have proven otherwise.
\vspace{0.2cm}
\newline
We did not build computers for the purpose of building a Mind. Computers, as I'm sure we are all aware, are calculating machines which can hold values in memory and usually
have persistent storage and some input/output devices. They were initially built to speed complex calculations, such as rapidly calculating the correct trajectory of a
projectile in a war zone, amongst other things. Prior to the advent of Electronic Computers, such calculations were performed by humans and checked by other humans which
was a time-consuming process. Today, a modest home computer can perform millions if not billions of these calculations in a single second. Obviously, there a no humans which
can match this speed of calculation.
\vspace{0.2cm}
\newline
A modern home computer may seem like much more than a calculator. Thanks to Operating Systems and Applications, we can use computers to play games, do work, socialise and 
countless other things. Connected to a high resolution display with a new AAA game loaded, what we experience almost seems like magic, but it is not. Everything from
Operating Systems to Applications to Games are built upon the computer performing rapid calculations and holding values in memory.
\vspace{0.2cm}
\newline
The reason for this simplification is to outline the domain in which we are attempting to build a Mind from an Electronic Computer. Although it seems computers are
capable of almost anything, most of what we experience as a computer user is illusions of things. We might see a button on the screen to click but the idea of a button
does not exist anywhere in the hardware of a computer. The pixels of the button are held in video memory, the action the button performs is controlled by the logic of
a program running on the operating system, and when we strip away all the layers, we arrive at mathematical operations which are performed on the CPU. Even something as
simple as a single character such as the letter 'A' does not exist anywhere in the computer hardware. Letters are simply represented by numeric values.
\vspace{0.2cm}
\newline
We now approach the first big question of this whole enterprise:
\begin{quote}
	\centering
	"Can a Mind exist in a Computer?"
\end{quote}
And, we do not yet know the answer. Let us then first clearly define what a computer is.

\section{What is a computer?}
Electronic Computers are built from transistors... Billions of them. So, what is it about a transistor that makes computers possible? 

\section{Unlimited Precision}


\chapter{Metaphysical Foundation}
\section{Intelligence}
This is some text....

\chapter{Untitled}




\end{document}
